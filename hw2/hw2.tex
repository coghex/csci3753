\documentclass[12pt, notitlepage, final]{article} 

\newcommand{\name}{Vince Coghlan}

%\usepackage[dvips]{graphics,color}
\usepackage{amsfonts}
\usepackage{amssymb}
\usepackage{amsmath}
\usepackage{latexsym}
\usepackage{enumerate}
\usepackage{amsthm}
%\usepackage{nccmath}
\usepackage{setspace}
\usepackage[pdftex]{graphicx}
\usepackage{epstopdf}
%\usepackage[siunitx]{circuitikz}
\usepackage{tikz}
\usepackage{float}
%\usepackage{cancel} 
\usepackage{setspace}
%\usepackage{overpic}
\usepackage{mathtools}
\usepackage{listings}
\usepackage{color}
\usepackage{qtree}
%\usepackage{gensymb}

\usetikzlibrary{calc}
\usetikzlibrary{matrix}
\usetikzlibrary{positioning}

\numberwithin{equation}{section}
\DeclareRobustCommand{\beginProtected}[1]{\begin{#1}}
\DeclareRobustCommand{\endProtected}[1]{\end{#1}}
\newcommand{\dbr}[1]{d_{\mbox{#1BR}}}
\newtheorem{lemma}{Lemma}
\newtheorem*{corollary}{Corollary}
\newtheorem{theorem}{Theorem}
\newtheorem{proposition}{Proposition}
\theoremstyle{definition}
\newtheorem{define}{Definition}
\newcommand{\column}[2]{
\left( \begin{array}{ccc}
#1 \\
#2
\end{array} \right)}

\newdimen\digitwidth
\settowidth\digitwidth{0}
\def~{\hspace{\digitwidth}}

\setlength{\parskip}{1pc}
\setlength{\parindent}{0pt}
\setlength{\topmargin}{-3pc}
\setlength{\textheight}{9.0in}
\setlength{\oddsidemargin}{0pc}
\setlength{\evensidemargin}{0pc}
\setlength{\textwidth}{6.5in}
\newcommand{\answer}[1]{\newpage\noindent\framebox{\vbox{{\bf CSCI 3753 Spring 2014} 
\hfill {\bf \name} \vspace{-1cm}
\begin{center}{Homework \#2}\end{center} } }\bigskip }

\DeclareMathOperator*{\argmin}{arg\,min}

%absolute value code
\DeclarePairedDelimiter\abs{\lvert}{\rvert}%
\DeclarePairedDelimiter\norm{\lVert}{\rVert}
\makeatletter
\let\oldabs\abs
\def\abs{\@ifstar{\oldabs}{\oldabs*}}
%
\let\oldnorm\norm
\def\norm{\@ifstar{\oldnorm}{\oldnorm*}}
\makeatother

\def\dbar{{\mathchar'26\mkern-12mu d}}
\def \Frac{\displaystyle\frac}
\def \Sum{\displaystyle\sum}
\def \Int{\displaystyle\int}
\def \Prod{\displaystyle\prod}
%\def \P[x]{\Frac{\partial}{\partial x}}
%\def \D[x]{\Frac{d}{dx}}
\newcommand{\PD}[2]{\frac{\partial#1}{\partial#2}}
\newcommand{\PF}[1]{\frac{\partial}{\partial#1}}
\newcommand{\DD}[2]{\frac{d#1}{d#2}}
\newcommand{\DF}[1]{\frac{d}{d#1}}
\newcommand{\fix}[2]{\left(#1\right)_#2}
\newcommand{\ket}[1]{|#1\rangle}
\newcommand{\bra}[1]{\langle#1|}
\newcommand{\braket}[2]{\langle #1 | #2 \rangle}
\newcommand{\bopk}[3]{\langle #1 | #2 | #3 \rangle}
\newcommand{\Choose}[2]{\displaystyle {#1 \choose #2}}
\newcommand{\proj}[1]{\ket{#1}\bra{#1}}
\def\del{\vec{\nabla}}
\newcommand{\avg}[1]{\langle#1\rangle}
\newcommand{\piecewise}[4]{\left\{\beginProtected{array}{rl}#1&:#2\\#3&:#4\endProtected{array}\right.}
\newcommand{\systeme}[2]{\left\{\beginProtected{array}{rl}#1\\#2\endProtected{array}\right.}
\def \KE{K\!E}
\def\Godel{G$\ddot{\mbox{o}}$del}

\onehalfspacing

\begin{document}

\answer{}

\begin{figure}[H]
\begin{center}
\includegraphics[width=14cm]{f1}
\end{center}
\end{figure}

A process context switch is when the OS takes the state of a process and stores in somewhere, then
restores another process state.  This allows for multitasking.  This occurs when the OS wants to
begin to execute another process while a current process is still running.  The entire process will
generally follow the following steps: first, a syscall or other interrupt that will enter the OS.
This syscall will then run and begin to store various things onto the OS's stack frame.  It will
then put another process's variables and such back on the stack, and return from the interrupt to
begin executing that code.

\begin{figure}[H]
\begin{center}
\includegraphics[width=14cm]{f2}
\end{center}
\end{figure}

Since processes require a full serperate address space, threads are much easier to create.  Since
threads share memory, however, one must be careful to modify the data one thread at a time, and this
can become costly for the program, and the programmer.  An advantage of a process is that a process
can run multiple threads within it, whereas a thread is just a thread.  Processes are much more simple
and far easier to understand for a typical programmer.  Threads require locking, mutexi, semaphores,
and are, overall, much more involved.

\begin{figure}[H]
\begin{center}
\includegraphics[width=14cm]{f3}
\end{center}
\end{figure}

Since much of the question isnt very specific, there will be $a$ women, $b$ men, and $c$
children trying to use the bathroom.  Assume that all of these people were declared in
either the init function, or somewhere else.  The details are in the comments, since much
of this code will not run.  Also, the problem contains inherent starvation, since if the
bathroom is full of women, and more women keep coming, the men will have to wait until
they are done.  There is not a way of fixing this without only letting a very small number
of a gender use the bathroom at a time and trade off, which is innefficient.  To solve this
problem I tried to make it as fair as possible, and in only very extreme cases will the
person waiting starve, or find another bathroom.

\begin{lstlisting}[language=C]
monitor bathroommonitor {
  // four states of the bathroom
  enum {mens, womens, empty, full} state;
  // each allows each person to be signaled.
  condition w[a];
  condition m[b];
  condition c[c];
  // number of men, women, children, and people currently
  // using the bathroom
  int men;
  int women;
  int children;
  int num;

  man_use_bathroom(int i) {
    // add man to list of people waiting to use the bathroom
    men++;
    // if there are women in there wait
    if (state == womens) {
      m[i].wait();
    }
    // if its full wait
    else if (state == full) {
      m[i].wait();
    }
    else if (state == (empty || mens)) {
      // go into the bathroom
      num++;
      // he is no longer waiting in line
      men--;
      // set the state to full if he makes it full
      if (num == N) {
        state = full;
      }
      else {
        state = mens;
      }
      // pee or whatever
      use_bathroom();
      // leave
      num--;
      // if he was the last person, signal the next women
      if (num == 0) {
        state = empty;
        if (women) {
          w[women].signal();
        }
      }
      else {
        state = mens;
        // if there are more children waiting than other guys,
        // let them go first
        if (children > men){
          c[children].signal();
        }
        else {
          m[men].signal();
        }
      }
    }
  }
  woman_use_bathroom(int i) {
    // this all works the same as with the men.
    women++;
    if (state == mens) {
      w[i].wait();
    }
    else if (state == full) {
      w[i].wait();
    }
    else if (state == (empty || womens)) {
      num++;
      women--;
      if (num == N) {
        state = full;
      }
      else {
        state = womens;
      }
      use_bathroom();
      num--;
      if (num == 0) {
        state = empty;
        if (men) {
          m[men].signal();
        }
      }
      else {
        state = womens;
        if (children > women) {
          c[children].signal();
        }
        else {
          w[women].signal();
        }
      }
    }

  }
  child_use_bathroom(int i) {
    children++
    if (state == full) {
      c[i].wait();
    }
    else {
      num++;
      children--;
      if (num == N) {
        state = full;
      }
      use_bathroom();
      num--;
      // signal the group that has the most people waiting for
      // fairness.
      if (men > women) {
        m[men].signal();
      }
    }
  }

  init() {
    state = empty;
    // in here somewhere we will set up all of the people who
    // want to use the bathroom, for now I will leave it general.
  }
}

\end{lstlisting}

\end{document}

